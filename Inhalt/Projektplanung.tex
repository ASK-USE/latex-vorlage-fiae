% !TEX root = ../Projektdokumentation.tex
\section{Projektplanung} 
\label{sec:Projektplanung}


\subsection{Projektphasen}
\label{sec:Projektphasen}

\begin{itemize}
	\item Das Projekt wurde im Zeitraum vom\\
	\textbf{15.11.2024 - 29.11.2024}\\
	unter Einsatz einer tägliche Arbeitszeit von\\
	\textbf{8 Stunden}\\
	erstellt.
    \item Die im Porjektantrag angefertigte Zeitplanung musst an mehereren Stellen angepasst werden aufgrund von Gründen welche an 
	gegebener Stelle näher erläutert werden. 
	 \textbf{!!!!!!!anpassung einfügen. gerne auch Bild einer Tabelle!!!!!!!!}

	\item Verfeinerung der Zeitplanung, die bereits im Projektantrag vorgestellt wurde.
\end{itemize}

\paragraph{Beispiel}
Tabelle~\ref{tab:Zeitplanung} zeigt ein Beispiel für eine grobe Zeitplanung.
\tabelle{Zeitplanung}{tab:Zeitplanung}{ZeitplanungKurz}\\
Eine detailliertere Zeitplanung findet sich im \Anhang{app:Zeitplanung}.


\subsection{Abweichungen vom Projektantrag}\label{sec:AbweichungenProjektantrag}
Auch bei der Durchführung des Projektes wurde von der im Antrag angegebenen Umsetzung an einigen Punkten abgewichen:\\
	\textbf{UML-Diagramme:} 
	bei den im Antrag angegebenen UML-Diagrammen wurde eine Satz an Diagrammen angekündigt, welcher der tatsächlichen Umsetzung nur bedingt genützt hätten. 
	weshalb folgendes Diagramme augetauscht wurden: 
	\begin{itemize}
		\item Klassendiagramm wurde durch Komponentendiagramm ersetzt
	Da Python zwar klassenorentiert konzipiert wurde, das Projekt jedoch keine Klassenobjekte benötigte, wurde ein ausführliches Komponentendiagramme erstellt.
		Dieses Diagramm wurde als geegneter angesehen die recht komplexen Abläufe und Verbindungen der Komponente untereinander zu veranschaulichen.  
		\item Aktivitätsdiagramm wurde zu drei Diagrammen ausgeweitet: 
		Aufgrund einer modularen Abtrennung der Komponenten untereinander wurde hier entschieden das ursprüngliche Diagramm um zwei weiter zu ergänzen. 
	\end{itemize}
	Die Gründe hierfür werden in der UML-Dokumentation näher erläutert\\

	\textbf{Architektur der Anwendung}
	Ursprünglich wurden im Projekt drei Module geplant: 
	\begin{itemize}
		\item main.py 
		\item com-server.py 
		\item test.db
	\end{itemize}
In der Umsetzung zeigte sich, das die main.py, welche als der Startpunkt aller Komponenten definiert wurde, unnötig war.
Da bei der Erstellung des für den Flaskserver benötigte Docker-Images ein Befehl zum ausführen der com_server.py-Datei integriert wurde und sonstige Startprozesse wie die Verknüpfung von Server mit Datenbank und das etablieren von Ports, Verbindungen und bereitstellen von API-Endpunkten vollständing zwischen Docker-Compose, 
Der Dockerfile und der REST-API realisiert wurde, hätte das erstellen einer externen, lokalen App eine Redundanz und unnötigen Ressourceneinsatz bedeutet.\\ 
	\textbf{Testphasen wurden Entsprechend an dem Veränderten Umsetzung angepasst.

. 


\subsection{Ressourcenplanung}
\label{sec:Ressourcenplanung}

\begin{itemize}
	\item Detaillierte Planung der benötigten Ressourcen (Hard-/Software, Räumlichkeiten \usw).
	\item \Ggfs sind auch personelle Ressourcen einzuplanen (\zB unterstützende Mitarbeiter).
	\item Hinweis: Häufig werden hier Ressourcen vergessen, die als selbstverständlich angesehen werden (\zB PC, Büro). 
\end{itemize}


\subsection{Entwicklungsprozess}
\label{sec:Entwicklungsprozess}
\begin{itemize}
	\item Welcher Entwicklungsprozess wird bei der Bearbeitung des Projekts verfolgt (\zB Wasserfall, agiler Prozess)?
\end{itemize}
