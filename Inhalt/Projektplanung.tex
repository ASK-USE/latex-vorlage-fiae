% !TEX root = ../Projektdokumentation.tex
\section{Projektplanung} 
\label{sec:Projektplanung}


\subsection{Projektphasen}
\label{sec:Projektphasen}

\begin{itemize}
	\item Das Projekt wurde im Zeitraum vom\\
	\textbf{15.11.2024 - 29.11.2024}\\
	unter Einsatz einer tägliche Arbeitszeit von\\
	\textbf{8 Stunden}\\
	erstellt.
    \item Die im Porjektantrag angefertigte Zeitplanung musst an mehereren Stellen angepasst werden aufgrund von Gründen welche an 
	gegebener Stelle näher erläutert werden. 
	 \textbf{!!!!!!!anpassung einfügen. gerne auch Bild einer Tabelle!!!!!!!!}

	\item Verfeinerung der Zeitplanung, die bereits im Projektantrag vorgestellt wurde.
\end{itemize}

\paragraph{Beispiel}
Tabelle~\ref{tab:Zeitplanung} zeigt ein Beispiel für eine grobe Zeitplanung.
\tabelle{Zeitplanung}{tab:Zeitplanung}{ZeitplanungKurz}\\
Eine detailliertere Zeitplanung findet sich im \Anhang{app:Zeitplanung}.


\subsection{Abweichungen vom Projektantrag}\label{sec:AbweichungenProjektantrag}
\begin{itemize}
    \item \textbf{UML-Diagramme:} Bei den im Antrag angegebenen UML-Diagrammen wurde eine Satz an Diagrammen angekündigt, welcher der tatsächlichen Umsetzung nur bedingt genützt hätte. Daher wurden folgende Diagramme augetauscht:
    \begin{itemize}
        \item \textbf{Klassendiagramm wurde durch Komponentendiagramm ersetzt:} Da Python zwar klassenorientiert konzipiert wurde, das Projekt jedoch keine Klassenobjekte benötigte, wurde ein ausführliches Komponentendiagramme erstellt. Dieses Diagramm wurde als geeigneter angesehen, um die recht komplexen Abläufe und Verbindungen der Komponenten untereinander zu veranschaulichen.
        \item \textbf{Aktivitätsdiagramm wurde zu drei Diagrammen ausgeweitet:} Aufgrund einer modularen Abtrennung der Komponenten untereinander wurde hier entschieden, das ursprüngliche Diagramm um zwei weitere zu ergänzen.
	\end{itemize}
Die Gründe hierfür werden in der UML-Dokumentation näher erläutert
\end{itemize}
\begin{itemize}
	\item\textbf{Architektur der Anwendung}
 Ursprünglich wurden im Projekt drei Module geplant:
	\begin{itemize}
    	\item main.py
    	\item com-server.py
    	\item test.db
	\end{itemize}
In der Umsetzung zeigte sich, dass die main.py, welche als der Startpunkt aller Komponenten definiert wurde, unnötig war. Da bei der Erstellung des für den Flaskserver benötigten Docker-Images ein Befehl zum ausführen der comserver.py-Datei integriert wurde und sonstige Startprozesse wie die Verknüpfung von Server mit Datenbank und das etablieren von Ports, Verbindungen und bereitstellen von API-Endpunkten vollständig zwischen Docker-Compose, Der Dockerfile und der REST-API realisiert wurde, hätte das Erstellen einer externen, lokalen App eine Redundanz und unnötigen Ressourceneinsatz bedeutet.}
\end{itemize}
\begin{itemize}
	\item \textbf{Testphasen wurden entsprechend an dem veränderten Umsetzung angepasst. Ebenso sind durch di veränderte Software die Diagramme nur noch bedingt brauchbar und 
eigentlich irreführen ohne weiterführenden Erklärung.}
\end{itemize}
 


\subsection{Ressourcenplanung}
\label{sec:Ressourcenplanung}

\begin{itemize}
	\item Detaillierte Planung der benötigten Ressourcen (Hard-/Software, Räumlichkeiten \usw).
	\item \Ggfs sind auch personelle Ressourcen einzuplanen (\zB unterstützende Mitarbeiter).
	\item Hinweis: Häufig werden hier Ressourcen vergessen, die als selbstverständlich angesehen werden (\zB PC, Büro). 
\end{itemize}


\subsection{Entwicklungsprozess}
\label{sec:Entwicklungsprozess}
\begin{itemize}
	\item Welcher Entwicklungsprozess wird bei der Bearbeitung des Projekts verfolgt (\zB Wasserfall, agiler Prozess)?
\end{itemize}
