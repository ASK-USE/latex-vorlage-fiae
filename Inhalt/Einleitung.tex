% !TEX root = ../Projektdokumentation.tex
\section{Einleitung}
\label{sec:Einleitung}


\subsection{Projektumfeld} 
\label{sec:Projektumfeld}
\textbf{Zum Praktikumsbetrieb:}\\
Das Fraunhofer Institut für Produktionstechnik und Automatisierung(IPA) - eines der größten Institute der Fraunhofer-Gesellschaft - wurde 1959 gegründet und beschäftigt rund 1200 Mitarbeiterinnen und Mitarbeiter. Organisatorische und technologische Aufgabenstellungen aus der Produktion bilden die Forschungs- und Entwicklungsschwerpunkte. Verfahren, Komponenten und Geräte bis hin zu kompletten Maschinen und Anlagen werden vom Institut entwickelt, erprobt und beispielhaft eingesetzt.\\
\\
\textbf{Zum Forschungsprojekt:}\\ Das Forschungsprojekt »H2GO – Nationaler Aktionsplan Brennstoffzellen-Produktion« des Fraunhofer Instituts hat das Ziel, die Produktion von Brennstoffzellen in Deutschland zu revolutionieren und damit einen wesentlichen Beitrag zur Reduzierung von CO2-Emissionen im Schwerlastverkehr zu leisten.\\
(Pressemitteilung H2Go: \href{https://www.iwu.fraunhofer.de/de/presse-und-medien/presseinformationen/PM-2022-ZV-Startschuss-fuer-das-Wasserstoffzeitalter-in-der-Lastenmobilitaet.html}{www.iwu.fraunhofer.de})\\
H2GO bündelt die Aktivitäten von 19 Fraunhofer-Instituten, um Technologien zu entwickeln, die eine kosteneffiziente und großflächige Produktion von Brennstoffzellen ermöglichen.\\
Das Projekt lässt sich in zwei Teilbereiche Aufteilen:
\begin{itemize}
	\item Produktion: auf vier Standorte verteilt, forscht an Umsetzungen und fertigen Komponenten für diese.
	\item Digitaliserung/Virtualisierung: Erstellt die digitalen und virtuellen Anteile der Referenzfabrik, koordiniert die Netzwerkstrukturen.
\end{itemize}
\textbf{Zum Teilbereich in welchem das Projekt realisiert wurde:}\\
Im Teilbereich der virtuellen Referenzfabrik wird die Digitalisierung und virtuellen Abbildung einer automatisierten Brennstoffzellenproduktion durch digitale 
Zwillinge nach \href{Navigation/DE/Home/home.html}{Industrie-4.0}-Standards umgesetzt.\\
Realisiert werden die digitalen Zwillinge dabei als Verwaltungsschalen (= AAS = Asset
Administration Shell).\\ 
 Aufgabe des Teilbereichs ist, eine möglichst exakte virtuelle Abbildung des gesamten Prozesses, des Produktes, der Produktionsdaten und -mittel, sowie Teilprodukte erzeugt werden.\\
 \\
\textbf{Auftraggeber:} \\
Auftragsvergabe für das REST-Frameworks erfolgte aufgrund meiner Überlegungen und Anregungen zur Ausfallsicherheit im Teilbereichs-Meeting und wurde durch Praktikums-Betreuer Dipl.-Ing Bumin Hatiboglu, wodurch dies ein internes Projekt des IPA und der betrefenden Abteilung ist.



\subsection{Projektziel}\label{sec:Projektziel}
\begin{itemize}
	\item \textbf{Was ist das Problem?}\\
    in der derzeitigen Planung und Umsetzung der Netzwerkarchitektur sollen die verschiedenen Daten des digitalen Zwillings von einem zentralen Registry-Server verwaltet werden, welcher die Speicherorte, Beziehungen und Verknüpfungen aller im Projekt vorhandenen Shells referenziert, und letztlich allen qualifizierten Clients ( Client = Stakeholder) bereit stellt.\\
    Dieser eine Registry-Server weist jedoch bisher keine wirkliche Ausfallsicherheit bei einem Netwerkausfall auf, was auf Basis einer Risikoanalyse besonders kritisch für die Ausfallsicherheit des Projektnetzwerk ist und bei Ausfall die Produktion letztlich zum Erliegen bringen würde.
	\item \textbf{Ziel des Projektes ist deshalb:}\\
    die Engstelle eines einzelnen Registryservers durch eine Redundantes System zu erweitern, was als Backupstrategie in der Regel  auch dem tatsächlich üblichen Industriestandard zur Riskiominimierung entspricht.  
    
\end{itemize}


\subsection{Projektbegründung}\label{sec:Projektbegruendung}
\begin{itemize}
	\item \textbf{Möglicher Benefit:}\\
	Abgesehen von dem offensichtlichen Vorteil einer Ausfallsicherheit bereits im reinen Forschungsbetrieb,
	ist es durch das Stadium der Umsetzung des H2Go-Projektes auch ein weitere, bereits sehr frühe implementierbare Verbesserung, und kann dadurch helfen Standards für spätere Umsetzungen zu 
	etablieren. Da in der Umsetzung nur kostenlose Open-Source Tools eingesetzt wurden, sorgt das Framework zudem für eine gleichzeitge Transparenz und Kostenersparniss.  
	\item \textbf{Motivation:}\\
	Motvation zur Erstellung des Projekts war, von Seiten des Durchführenden, in seiner Projektarbeit einerseits eine gut umsetzbare Lösung zu präsentieren. Anderseits war eine Motiv dahinter die Thematik einer 
	robusten Infrastruktur bereits auf dieser Ebene zu etablieren um dazu beizutragen unserer sich ständig wachsenden digitalen Welt stabile Systeme zugrunde zu legen. 

\end{itemize}


\subsection{Projektschnittstellen} 
\label{sec:Projektschnittstellen}
\begin{itemize}
	\item Die Anwendung soll zukünftig mit einer weiteren REST-API aus dem AAS-Registry-Server des \href{https://eclipse.dev/basyx/}{Eclipse-BaSyx-Projektes} 
	verbunden werden, jedoch ist dies noch nicht implementiert worden, da diese Strukturen gerade im Aufbau sind. 
	Als weiters System mit welchem die Anwendung aktuell bereits verbunden und getestet wurde, darf man wohl den als Teil des Projekts erstellten Locust-Lasttest 
	verstehen. Dieser hilft, Tests der Auslastung des Systems durchzuführen.
	\item Da dieses Projekt Teil eines bereits genehmigten Forschungsprojektes ist, wird es durch das Förderprogramm "go-digital" finanziert, das vom 
	Bundesministerium für Wirtschaft und Klimaschutz (BMWK) ins Leben gerufen wurde.   
	\item im weitern Verlauf des Projektes muss das Framework von Anwendungsentwicklern der Fraunhofer weiterentwickelt und angepasst werden und von diesen, oder den
	Verantwortlichen der jeweieligen am H2Go-Projekt teilnehmenden Instituten installiert und gewartet werden. Jedoch besteht hier weder eine konkrete Umsetzung noch 
	eine konkrete Planung.  
	\item Das Projekt wurde den verantwortlichen Betreuern vorgelegt und wurde vom Softwareverantwortlichen Sven Lieckfeld bereits einer Endabnahme unterzogen. Zum 
	aktuellen Zeitpukt konnte die Endabnahme duch den Projektverantwortlichen Bumin Hatiboglu aus gesundheitlichen Gründen noch nicht durchgeführt werden.   
\end{itemize}


\subsection{Projektabgrenzung} 
\label{sec:Projektabgrenzung}
\begin{itemize}
	\item Das Projekt soll explizit nicht als fertig entwickelte und einsatzfähige synchrone Backuplösung für den Registry-Server verstanden werden, da keine 
	derartigen Funktionalitäten implementiert oder Integriert wurden. Das Projekt stellt lediglich ein Framework zur Verfügung, auf wlechem eine solche Lösung 
	Implementiert werden kann. Es sollte somit eher als Werkzeug und spezialisierte, aber nicht spezifizierte Schnittstelle verstanden werden.   
\end{itemize}
